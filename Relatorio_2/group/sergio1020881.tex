\section{Plano de desenvolvimento pessoal de competências}
\qquad O meu plano de desenvolvimento pessoal, passa por obter mais formação e aprender com pessoas com mais experiência em diversas áreas, que é exatamente o que estou a fazer frequentando o curso de Engenharia Eletrotécnica e de Computadores no \textcolor{gray}{I.S.E.P}.\\
Esta disciplina em particular é uma forma de poder enriquecer minhas competências e metodologias de Gestão, e perceber as restantes matérias abordadas que compõem o \textcolor{blue}{Comportamento Organizacional}.\\
\begin{minipage}{10.3cm}
\begin{figure}[H]
	%\flushleft
	\begin{minipage}{\linewidth}
		\flushleft
		\includegraphics[scale=0.3]{"./image/Skills/Managerial Skills for the Global Marketplace.jpg"}
	\end{minipage}
	\caption{Competências de Gestão. \cite{book_6}}
\end{figure}
\end{minipage}
\begin{minipage}{12cm}
	\textbf{Metodologias da gestão}: \cite{book_9}\\
	\\
	\begin{minipage}{3.1cm}
		Instrumental
		\begin{enumerate}
			\setlength\itemsep{-0.3em}
			\item Planear
			\item Organizar
			\item Controlar\\
		\end{enumerate}
	\end{minipage}
	\begin{minipage}{5cm}
		Comportamental
		\begin{enumerate}
			\setlength\itemsep{-0.3em}
			\item Liderança
			\item Comunicação
			\item Motivação
			\item Tomada de decisão
		\end{enumerate}
	\end{minipage}
\end{minipage}
\\
\\
No entanto por enquanto minha missão é concluir a formação, e ao mesmo tempo melhorar um conjunto de ferramentas e métodos de trabalho para que seja estável e eficaz de forma a poder resolver os problemas que possa ter que enfrentar com facilidade, e eventualmente realizar alguns projetos pessoais. \\
\\
\newpage
\subsection{Análise S.W.O.T Pessoal}
\qquad Neste contexto de plano de desenvolvimento a análise \textcolor{blue}{SWOT} também pode ser uma ferramenta útil de forma a nos indicdar qual os comportamentos que poderá ser melhorado ou alterado, com a ajuda de terceiros talvez é mais eficaz. \\

\begin{itemize}
	\setlength\itemsep{-0.85em}
	\item \textcolor{purple}{I}nterno
	\begin{itemize}
		\setlength\itemsep{-0.3em}
		\item \textcolor{orange}{S}trength (forças) \\
		- Numeração, Literacia Bilingue, Resolução de Problemas \\
		- Formação Académica, Experiência Profissional \\
		- Facilidade de Adaptação e aprendizagem, Imaginação, Estabilidade Emocional \\
		- Gestão e Comunicação, Organização Pessoal, Numeração Avançada, Tecnologias de informação e comunicação. \\
		- Empatia, Método Cientifico
		\item \textcolor{orange}{W}eakness (fraquezas) \\
		- Contabilidade e Vendas \\
		- Direto, Crítico, Detesto desigualdade e injustiças \\
		- Frontal com contradições \\
		- Só respeito quem merece, ou tem atributos positivos \\
		- "Dente por dente e olho por olho" \\
		- anti-Dogma
	\end{itemize}
	\item \textcolor{purple}{E}xterno
	\begin{itemize}
		\setlength\itemsep{-0.3em}
		\item \textcolor{orange}{O}pportunity (oportunidades) \\
		- Nenhum
		\item \textcolor{orange}{T}hreats (ameaças) \\
		- Cultura Portuguesa \\
		- Sistema Político-Social \\
		- Racismo
	\end{itemize}
\end{itemize}
Algumas explicações de personalidade descrevo no caso de quando se diz "dente por dente e olho por olho", muitas das vezes tem interpretação errada, pois concluem que existiria apenas cegos após alguns tempos, mas sendo uma metáfora, sabe-se que ninguém vai andar a cegar uns aos outros sem motivo e são circunstancias de saber individual, mas deve ser percebido no aspeto em que uma pessoa que é honesta merece honestidade, e uma humilde humildade, e pelo verso um mentiroso aldrabado, e assassino deve ser morto, este procedimento leva com que o bem vence sempre, isto é lógico e citações milenares de certa forma condiz neste caso. Que levanta também a questão da veracidade da perceção, na qual muito cuidado é exigido. \\
Dai que certas pessoas quando estão a ser irónicas, acabam dececionados com as reações esperadas, podendo entrar em ciclos viciosos que só vão agravando, lol.\\
O método cientifico nos diz que se um acontecimento se repete nas mesmas circunstancias e nunca se altera é considerado facto ou lei ou teoria, é uma arte de reconhecer padrões. Também nós ensina que os conhecimentos estão sempre abertos ao escrutínio e se houver prova que refuta a teoria esta deixa de o ser, ou seja, é tentar representar a realidade observada por modelos racionais e matemáticos as ferramentas que estão ao nosso dispor, já que não existe melhor. \\

\subsection{Curriculum Vitae}
\qquad





\begin{comment}
a) Realizar um diagnóstico de competências pessoais;\\
b) Definir objetivos de carreira;\\
c) Definir as competências que considere que no futuro lhe permitirão atingir os referidos objetivos;\\
d) Definir um plano de desenvolvimento para as competências anteriormente selecionadas.\\
Eliminate waste, reduce errors and improve.
\end{comment}