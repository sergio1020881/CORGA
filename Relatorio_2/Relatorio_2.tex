\input{./input/PREAMBLEREPORT}
\begin{minipage}{\linewidth}

\title{Comportamento Organizacional}
\author{
\emph{S\'{e}rgio Santos},\;$N^o$:\; 1020881 \\
\emph{Nome 2},\;$N^o$:\; 2000000\\
\emph{Nome 3},\;$N^o$:\; 3000000\\
%\emph{Nome 4},\;$N^o$:\; 4000000\\
%\emph{Nome 5},\;$N^o$:\; 5000000\\
}
\date{\today}
%\titlepic{\includegraphics[scale=0.50]{./image/ROQ/ROQ.jpg}}

\begin{titlepage}
\includegraphics[scale=0.60]{./image/capa/ISEP_marca_cor_grande.png}
\maketitle
\vspace{8cm}
\begin{flushleft}
\includegraphics[scale=0.50]{./image/ROQ/ROQ.jpg}
\end{flushleft}
\end{titlepage}

\end{minipage}

\tableofcontents
\appendix
\pagestyle{plain} %plain headings empty
%\setcounter{chapter}{0}
%\numberwithin{page}{section}
%\renewcommand{\abstractname}{Executive Summary}
\captionsetup{justification = raggedright, singlelinecheck = false}
\setlength{\parindent}{0in}
%%%%%%%%%%%%%%%%%%%%%%%%%%%%%%%%%%%%%%%%%%%%%%%%%%%%%%%%%%%%%%%%%%%%%%%%%%%%%%%%%%%%%%%%%%%%%%%%%%%%%%%%%%%%%%%
\label{Resumo}
\begin{abstract}
\qquad Este trabalho vai abordar os assuntos mencionados no titulo, primeiro vai ser classificado as classes de trabalhador e uma breve explicação e comparação simples das mais comum na nossa sociedade o trabalhador por conta de outrem e o independente, uma breve descrição das novas tendências no mercado de trabalho que estão a surgir, também mostrar a importância da gestão da carreira individual e formas de a valorizar.\\
\\
Vai ser também abordado a importância da mudanças com a entrada das novas tecnologias no ambiente de trabalho, um estudo das competências e a valorização pessoal.\\
\\
A matéria relacionada com esta disciplina estará sempre presente nos assuntos abordados.\\
\vfill
\textbf{Palavras Chave:} Mudança, Liderança, Gestão de Equipas, ICT
\end{abstract}
%%%%%%%%%%%%%%%%%%%%%%%%%%%%%%%%%%%%%%%%%%%%%%%%%%%%%%%%%%%%%%%%%
\newpage
\section{Introdução}
%%%%%%%%%%%%%%%%%%%%%%%%%%%%%%%%%%%%%%%%%%%%%%%%%%%%%%%%%%%%%%%%%
\qquad A juventude foi ensinada que a missão de se tornarem adultos, o caminho de dignidade, segurança e independência é obter um emprego.\cite{book_11}\\
O estágio uma ferramenta muito importante para qualquer iniciante de qualquer profissão de forma a ser transmitido conhecimentos adquiridos, que em certas profissões pode demorar até alguns anos para alcançar a categoria de oficial ou sénior.\\
\\
O cidadão pode estar na situação contributiva abaixo descrito para ser considerado um trabalhador em regime legal.\\
\\
\begin{minipage}[t]{\linewidth}
\begin{itemize}
\setlength\itemsep{-0.3em}
\item Trabalhador por conta de outrem
\begin{itemize}
\item Organização privada
\item Organização pública
\end{itemize}
\item Trabalhadores independentes
\begin{itemize}
\item Recibos Verdes
\item Falsos Recibos Verdes
\item subcontratado
\end{itemize}
\newpage
\item Trabalhador do serviço domestico
\item Membros de órgãos estatuários
\item Empresa
\item Político\\ \\
\end{itemize}
\end{minipage}
\subsection{trabalhador por conta de outrem}
\begin{figure}[H]
%\centering
\flushleft
%\includegraphics[width=.6\textwidth,left]{./image/SGS/Contribuicoes_1.jpg}
\includegraphics[scale=.5]{./image/SGS/Contribuicoes_1.jpg}
\caption{Contribuições para SGS}
\end{figure}\par
Como podemos ver o cidadão desconta, 34,5\% e 33,3\% respetivamente para o estado.\\
Exemplo:\\ 
Vencimento de 1000Eur será descontado 11\% para a Segurança Social, ficando com $1000\times (1-0,11)=890Eur$ e a empresa desconta para o exemplo de 23,75\%, $1000\times 0,2375=237,5Eur$, ao todo será descontado $110+237,5=347,5Eur$, ou seja, todos os messes um trabalhador que ganhe 1000Eur desconta para a Segurança Social direto e indiretamente \textbf{347,5Eur}.\\ \\
Na realidade o vencimento neste exemplo do cidadão é de \textbf{1237,5Eur}, ou seja, é prejudicado nos seus descontos na quantia de 237,5Eur [23,75\%] pois não são considerados como pessoais.\\
A circulação deste capital passa despercebido e usado pelo estado para seus gastos, sendo o cidadão sua fonte, sem nenhum proveito, a não ser as empresas que depois recebem ajudas através desta receita.\\ \\
Em Geral a receita laboral de um cidadão é quase três oitavos $23,75\%+11\%=34,75\%$ depois dos respetivos descontos [1000Eur \textit{vs} 347,5Eur].\\ \\
Estas contas são feitas sem considerar qualquer subsidio de alimentação.
\subsection{Trabalhadores Independentes}
%%%%%%%%%%%%%%%%%%%%%%%%%%%%%%%%%%%%%%%%%%%%%%%%%%%%%%%%%%%%%%%%%
Este tipo de trabalhador em principio pode definir seus descontos numa dada margem, e é aliciante para as empresas este tipo de trabalhador pois não tem qualquer responsabilidade, este acarreta toda a responsabilidade de descontos e despesas, no entanto em principio ira ganhar mais do que o trabalhador por conta de outrem, mas descontando muito menos e prejudicado a longo prazo devido a concorrência, a não ser que desconte a totalidade de $23,75\%+11\%=34,75\%$ e ainda obter um vencimento superior ao seu equivalente de trabalhador por conta de outrem.
\subsection{Precariedade}
%%%%%%%%%%%%%%%%%%%%%%%%%%%%%%%%%%%%%%%%%%%%%%%%%%%%%%%%%%%%%%%%%
Nenhum cidadão deve aceitar qualquer trabalho que ganhe menos que \; $ \mbox{\Large $ \frac{635Eur}{0,65}\approx 977Eur $ } $ para se dizer que leva uma vida sustentável, pois o salário mínimo nacional é de 635Eur, e se ficar em \textit{lay off} ou \textit{desempregado}, como demonstrado:\\
$635\times(1-0,11)\approx566Eur$,\\
$635\times(0,3475)\approx220Eur$,\\
$\frac{635\times14}{12}\times0,65 \approx 482Eur$, \\
estará a trabalhar gratuitamente, só ira receber \textbf{566Eur} com descontos de \textbf{220Eur}, ou seja um escravo do estado. No caso de \textit{lay-off ou desemprego} recebera apenas 482Eur.\\
Em principio qualquer remuneração será deduzido por: $Vencimento \times (1-0,11) \times (1-0,23) \times - Combustivel\times 0,61 = Rendimento \, Líquido$ pois tudo também leva IVA e taxa de combustível.\\
ex: (\textit{individuo com salario mínimo nacional})\\
1. vencimento = 635Eur e 0Eur gasolina mensal \\
\hspace*{1cm} $635Eur \times (1-0,11) \times (1-0,23) \times - 0Eur \times 0,61 = 435Eur$ \\
2. vencimento = 635Eur e 80Eur gasolina mensal \\
\hspace*{1cm} $635Eur \times (1-0,11) \times (1-0,23) \times - 80Eur \times 0,61 = 386Eur$ \\
3. vencimento = 635Eur e 150Eur gasolina mensal \\
\hspace*{1cm} $635Eur \times (1-0,11) \times (1-0,23) \times - 150Eur \times 0,61 = 343Eur$, \\ \\
mas ainda não acaba aqui a pintura negra, supondo agora que o cidadão não tem caro, ou seja, recebe limpos 435Eur, ainda vai ter que pagar taxa água e saneamento (mínimo 11,3Eur) e taxa de luz (mínimo 8Eur). Fica com 415,7Eur para piorar vamos supor que tem habitação e tem que pagar IMI (mínimo 11Eur/mês).
Se este exemplo tiver um empréstimo de habitação e ou um veiculo chegamos a conclusão que não pode se alimentar, o que será muito bom para a dieta, e doenças.\\
\newpage
Concluindo que no estado presente de trabalho só é benéfico se pertencermos aos membros de órgãos estatuários ou político, pois não tem encargos do estado e aufere de regalias e vencimentos mínimo de cinco vezes e até dez vezes superior ao salário mínimo nacional, também existindo casos excecionais de vinte e para cima a mais o salário mínimo nacional. Sendo que esta profissão existe apenas por tráfico de influências e não igualdade ou equidade, e muito menos competência.
\subsection{mudança}
%%%%%%%%%%%%%%%%%%%%%%%%%%%%%%%%%%%%%%%%%%%%%%%%%%%%%%%%%%%%%%%%%
Já é conhecido que em \textsf{2025}, 75\% da classe trabalhadora vai pertencer a geração \textbf{Z}, e o quadro do futuro de trabalho esta cada vez mais centrado a volta do desenvolvimento tecnológico, as sociedades vão ter que o acompanhar o ritmo de crescimento, e a União Europeia e seus membros reconhecem esta tendência e a necessidade de formação e treino destas competências nos trabalhadores Europeus, sendo o projeto \textit{industria 4.0} uma destas ferramentas.\\
\\
Empresas de todo tipo e dimensão estão a ser enfrentados com a questão de como podem assegurar o fornecimento de lideres com as competências, habilidades e visão estratégica adequadas para obter o sucesso. Ignorando a velha mentalidade de que certos indivíduos nascem para liderar, muitas empresas acreditam que a liderança pode ser desenvolvida numa forma pro-ativa e de forma sistemática.\cite{book_6}
\section{O futuro do trabalho}
%%%%%%%%%%%%%%%%%%%%%%%%%%%%%%%%%%%%%%%%%%%%%%%%%%%%%%%%%%%%%%%%%
\qquad Agora com as novas tecnologias tem se aberto várias portas para novas formas de as pessoas serem remuneradas por seus serviços ou bens. Exemplos muito notórios são casos como a UBER, AMAZON, YOUTUBE, LINKEDIN, etc.\\

Esta a fugir para uma forma de trabalhadores independentes, controlado por sistemas tecnológicos administrativos, empresas virtuais, que pode ser formas de exploração e concorrência desleal, quando mal usados, que proporcionam enriquecimento rápido aos que implementem estes sistemas e o gerem.

However, the future of work is changing from a directive approach towards collaborative frameworks that cause employees to think and act differently.

The more I learned, the more I understood that the startup “future of work” story, as consoling as it was, was also incomplete. Yes, the gig economy could create opportunity for some people, but it also could amplify the same problems that made the world of work look so terrifying in the first place: insecurity, increased risk, lack of stability, and diminishing workers’ rights. The gig economy touched many people. Some of them were rich, some poor, some had power, and some didn’t. Its impact on each of them was different.


\newpage
\section{A gestão de carreira e as competências necessárias num mundo em mudança}
%%%%%%%%%%%%%%%%%%%%%%%%%%%%%%%%%%%%%%%%%%%%%%%%%%%%%%%%%%%%%%%%%
\input{./group/Competencias}

Trabalho especializado ou de rotinas.\\

shows that workers in digital-intensive industries on average exhibit higher levels of cognitive as well as non-cognitive skills and social skills than workers in less digitally-intensive sectors of the economy. This may of course depend on the type of workers employed in digital vs less digital intensive sectors, with the latter being generally more intensive in unskilled workers.\\

Skills are better rewarded in digital intensive sectors\\

Table 2 shows that for two types of skills, labour market returns are higher in digital intensive industries than in less digital intensive industries. These are advanced numeracy skills and self-organisation skills.\\

these results are robust to a number of checks. In Table A2, each interaction term enters the specification alone (rather than introducing all interaction terms at the same time, as in Table 2). The results show that all type of skills are better rewarded in digital intensive industries, in particular advanced numeracy skills, self-organisation skills, ICT skills and numeracy. However, in these specifications the estimated coefficients for the skill variables are clearly biased upwards by the omission of all other skill variables, which are correlated with the individually included skill.\\



Wage returns to ICT skills are twice as big as those related to numeracy skills, whereas management and communication skills are rewarded as much as numeracy skills are. These results also indicate the importance for workers to adequately use their skills and human capital on the job. This in turns calls for workers to be well matched to tasks they perform on the job and stresses the importance of good human resource and organisational management within firms.

the returns to task-based skills are still in the same ballpark than those for cognitive skills constitutes strong evidence about the importance of task-specific human capital for production and labour markets. Wage returns to ICT skills are twice as big as those related to numeracy skills, whereas management and communication skills are rewarded as much as numeracy skills are. These results also indicate the importance for workers to adequately use their skills and human capital on the job. This in turns calls for workers to be well matched to tasks they perform on the job and stresses the importance of good human resource and organisational management within firms.\\

Workers in digital intensive sectors are generally better paid\\

Only Deming (2015), who uses data on cognitive and non-cognitive skills for workers in the U.S., finds that in ICT intensive jobs bundles of cognitive and non-cognitive skills are better rewarded.\\

Especially in digital intensive industries, the production process might be more decentralised (as e.g. outsourcing, offshoring and vertical integration may occur more frequently) and the bundling of self-organisation and management and communication with advanced numeracy skills is key to ensure the functioning of production and the workflow within teams.\\

The findings indicate that cognitive as well as non-cognitive skills are strongly rewarded by labour markets, even when controlling for industry and occupation fixed effects, individual control variables (including education) and firm size. In addition, the fact that the returns to task-based skills are still in the same ballpark than those for cognitive skills highlights the importance of task- specific human capital for production and labour markets. Wage returns to ICT skills are twice as large as those to numeracy skills, whereas management and communication skills are almost equally rewarded as numeracy skills.\\

Furthermore, the results show that digital intensive industries reward workers having relatively higher levels of self-organisation and advanced numeracy skills more than less digital intensive industries. Moreover, for workers in digital intensive industries, bundles of skills are particularly important: workers endowed with a high level of numeracy skills receive an additional wage premium, if they also show high levels of self-organisation or managing and communication skills. Thus, the present analysis contributes to inform policy makers on the relationship between the digital transformation and skills’ needs and supply. This is essential for the design of effective and forward-looking skills and employment policies, aimed at aligning labour market demand and supply and at fostering productivity.\\

Understanding which skills, cognitive as well as non-cognitive, yield high returns in the labour market is also important to address inequality issues and to foster employment and well-being. If the demand of certain skills and certain bundles of skills outgrow the supply for such skills, the rewards of these skills would increase while those for other skills would decline. This skill shortage could easily lead to rising wage inequality and even to a surge of unemployment of workers not possessing these types of skills. For this reason, to curb the rising wage inequality, it may be important to design or target training programs so that they better prepare workers for the specific skills being high in demand, given the acceleration of the digital transformation across occupations and industries. Moreover, the earlier the training takes place, the lower the costs of trainings for cognitive as well as non-cognitive skills become (Cunha and Heckman, 2007; Cunha et al., 2010). Therefore, recognising major skills needed for digitalisation is important not only for labour market policy interventions, but also for policies specifically targeting the education sector.\\

Finally, as the digital transformation will soon also affect industries that are at present less digitalised, governments will more and more need to equip their populations with a wide range of skills, and to continue doing so over time. This might entail not only strengthening quantitative or cognitive skills, but also combining these with a good endowment of non-cognitive and socio-emotional skills. The proposed work can therefore help countries identifying the sets of key skills their citizens need to be equipped with to succeed in the digital era. This is important for the design and implementation of education and training programmes, as well as to enhance labour market participation and workers’ performance.\\

The research on personality traits (or non-cognitive skills), especially for North America, has been organised following the so-called "Big Five" factor model of personality (Goldberg, 1990). This suggests that most personality measures could be subsumed under an umbrella including five key factors:
extraversion, agreeableness, conscientiousness (or dependability), emotional stability (vs. neuroticism), and openness to experience.\\







\section{plano de desenvolvimento pessoal de competências}
%%%%%%%%%%%%%%%%%%%%%%%%%%%%%%%%%%%%%%%%%%%%%%%%%%%%%%%%%%%%%%%%%
\textbf{Mário Marante}\\
plano de desenvolvimento pessoal de competências\\
\\ \\
\textbf{João Pereira}\\
plano de desenvolvimento pessoal de competências\\
\\


ola sou joão.\\ \\
\textbf{Sérgio Santos}\\
O meu plano de desenvolvimento pessoal, passa por obter mais formação e apreender com pessoas com mais experiência em diversas áreas, que é exatamente o que estou a fazer frequentando o curso de Engenharia Eletrotécnica e de Computadores no \textcolor{gray}{I.S.E.P}.\\
Esta disciplina em particular é uma forma de poder enriquecer minhas competências de Gestão, e as matérias abordadas que compõem o \textcolor{blue}{Comportamento Organizacional}, que são bastantes.
\begin{figure}[H]
	%\centerline
	\begin{minipage}{\linewidth}
		\centering
		\includegraphics[scale=0.3]{"./image/Skills/Managerial Skills for the Global Marketplace.jpg"}
	\end{minipage}
	\caption{Competências de Gestão. \cite{book_6}}
\end{figure}






\begin{comment}
a) Realizar um diagnóstico de competências pessoais;\\
b) Definir objetivos de carreira;\\
c) Definir as competências que considere que no futuro lhe permitirão atingir os referidos objetivos;\\
d) Definir um plano de desenvolvimento para as competências anteriormente selecionadas.\\
Eliminate waste, reduce errors and improve.
\end{comment}\\ \\



que o aluno considere

fundamentais no contexto do trabalho do futuro, para atingir os seus objetivos
profissionais. Para tal, deverá:
a) Realizar um diagnóstico de competências pessoais;
b) Definir objetivos de carreira;
c) Definir as competências que considere que no futuro lhe permitirão atingir os
referidos objetivos;
d) Definir um plano de desenvolvimento para as competências anteriormente
selecionadas.
• A componente individual do trabalho a realizar por cada aluno deverá ser entregue em forma de um padlet (https://padlet.com) que demonstre todas as pesquisas e inputs a que o aluno recorreu para a realização do trabalho bem como as suas reflexões. O padlet deve terminar com a apresentação do plano de
desenvolvimento pessoal de competências.
• Cada aluno deverá defender oralmente a componente individual.


\newpage
\section{Conclusão}
%%%%%%%%%%%%%%%%%%%%%%%%%%%%%%%%%%%%%%%%%%%%%%%%%%%%%%%%%%%%%%%%%



%%%%%%%%%%%%%%%%%%%%%%%%%%%%%%%%%%%%%%%%%%%%%%%%%%%%%%%%%%%%%%%%%
\newpage
%%%%%%%%%%%%%%%%%%%%%%%%%%
%\input{./input/EQUACAO}
%\input{./input/DEFINICAO}
%%%%%%%%%%%%%%%%%%%%%%%%%%
%Figuras Bibliografia Index
\listoffigures
\cite{*}
\bibliography{./bibliography/Bibliography}
%\printindex
\newpage
\footnote{Apontamento}
\end{document}
%%%%%%%%%%%%%%%%%%%%%%%%%%%%%%%%%%%%%%%%%%%%%%%%%%%%%%%%%%%%%%%%%
\begin{comment}
Vamos montar uma empresa, eu monto e tu vais ser a empresa. \\
Um palerma nasce a cada minuto.\\
No futuro existe um grau elevado de incerteza, com muitos problemas pelo caminho, a solução para os problemas tem duas soluções, ou fazer nada, na qual tem duas saídas, ou fica estático, isto é igual mas com um grande grau de probabilidade de piorar, a outra alternativa é fazer mudanças, seja qual for e esperar o resultado, caso negativo mudar novamente de direção desde que não seja a mesma, a rapidez de este processo tem uma grande esperança de garantia de sucesso, no entanto haverá casos de problemas já reconhecidos e os remédios também, dando um avanço e vantagem a nível de competição, isto requer sabedoria e conhecimentos adquiridos das velhas gerações.\\
Existe a preocupação da desigualdade da distribuição da riqueza sendo que a procura vai estar mais centrada a volta de competências tecnológicas e existir disparidade salarial entre as classes trabalhadoras, no entanto Portugal vive numa cultura antitético a estas mudanças com lideres totalmente desligados e desatualizados, e quando o país atravessa situações de calamidade a solidariedade são suas únicas formas de luta.\\
\end{comment}
%%%%%%%%%%%%%%%%%%%%%%%%%%%%%%%%%%%%%%%%%%%%%%%%%%%%%%%%%%%%%%%%%