\input{./input/PREAMBLEREPORT}
\begin{minipage}{\linewidth}

\title{Comportamento Organizacional}
\author{
\emph{S\'{e}rgio Santos},\;$N^o$:\; 1020881 \\
\emph{Nome 2},\;$N^o$:\; 2000000\\
\emph{Nome 3},\;$N^o$:\; 3000000\\
%\emph{Nome 4},\;$N^o$:\; 4000000\\
%\emph{Nome 5},\;$N^o$:\; 5000000\\
}
\date{\today}
%\titlepic{\includegraphics[scale=0.50]{./image/ROQ/ROQ.jpg}}

\begin{titlepage}
\includegraphics[scale=0.60]{./image/capa/ISEP_marca_cor_grande.png}
\maketitle
\vspace{8cm}
\begin{flushleft}
\includegraphics[scale=0.50]{./image/ROQ/ROQ.jpg}
\end{flushleft}
\end{titlepage}

\end{minipage}

\tableofcontents
\appendix
\pagestyle{plain} %plain headings empty
%\setcounter{chapter}{0}
%\numberwithin{page}{section}
%\renewcommand{\abstractname}{Executive Summary}
%\captionsetup{justification = raggedright, singlelinecheck = false}
\setlength{\parindent}{0in}
%%%%%%%%%%%%%%%%%%%%%%%%%%%%%%%%%%%%%%%%%%%%%%%%%%%%%%%%%%%%%%%%%%%%%%%%%%%%%%%%%%%%%%%%%%%%%%%%%%%%%%%%%%%%%%%
\label{Resumo}
\begin{abstract}
\qquad Este trabalho vai abordar os assuntos mencionados no titulo, primeiro vai ser classificado as classes de trabalhador e uma breve explicação e comparação simples das mais comum na nossa sociedade, o trabalhador por conta de outrem e o independente, uma breve descrição das novas tendências no mercado de trabalho que estão a surgir, também mostrar a importância da gestão da carreira individual e formas de a valorizar. \\
\\
Vai ser abordado a importância da mudanças com a entrada das novas tecnologias no ambiente de trabalho, e um estudo das competências. \\
\\
É para realçar a adaptação e mudança dos comportamentos dos líderes, e o adquirir das competências necessárias por parte de todos. As Organizações devem evoluir e desenvolver de forma a poder servir o mercado e a sociedade em geral, na qual sua sobrevivência dependente cada vez mais.
\vfill
\textbf{Palavras Chave:} Mudança, Liderança, Gestão de Equipas
\end{abstract}
%%%%%%%%%%%%%%%%%%%%%%%%%%%%%%%%%%%%%%%%%%%%%%%%%%%%%%%%%%%%%%%%%
\newpage
\section{Introdução}
%%%%%%%%%%%%%%%%%%%%%%%%%%%%%%%%%%%%%%%%%%%%%%%%%%%%%%%%%%%%%%%%%
\qquad A juventude foi ensinada que a missão de se tornarem adultos, o caminho de dignidade, segurança e independência é obter um emprego. \cite{book_11} \\
E o estágio uma ferramenta muito importante para qualquer iniciante de qualquer profissão de forma a ser transmitido conhecimentos adquiridos, que em certas profissões pode demorar até alguns anos para alcançar a categoria de oficial ou sénior. \\
\\
O cidadão deve estar numa das situações contributiva abaixo descrito para ser considerado um trabalhador em regime legal. \\
\\
\begin{minipage}[t]{\linewidth}
\begin{itemize}
\setlength\itemsep{-0.3em}
\item Trabalhador por conta de outrem
\begin{itemize}
\item Organização privada
\item Organização pública
\end{itemize}
\item Trabalhadores independentes
\newpage
\item Trabalhador do serviço domestico
\item Membros de órgãos estatuários
\item Empresa
\item Político
\end{itemize}
\end{minipage}
\subsection{Trabalhador por conta de outrem}
\begin{figure}[H]
%\centering
\flushleft
%\includegraphics[width=.6\textwidth,left]{./image/SGS/Contribuicoes_1.jpg}
\includegraphics[scale=.5]{./image/SGS/Contribuicoes_1.jpg}
\caption{Contribuições para SGS}
\end{figure}\par
\textbf{Exemplo:} \\
\\
Vencimento de 1000Eur será descontado 11\% para a Segurança Social, ficando com $1000\times (1-0,11)=890Eur$ e a empresa desconta para o exemplo de 23,75\%, $1000\times 0,2375=237,5Eur$, ao todo será descontado $110+237,5=347,5Eur$, ou seja, todos os messes um trabalhador que ganhe 1000Eur desconta para a Segurança Social direto e indiretamente \textbf{347,5Eur}. \\ \\
Na realidade o vencimento neste exemplo do cidadão devia ser de \textbf{1237,5Eur}, ou seja, é prejudicado nos seus descontos na quantia de 237,5Eur [23,75\%] pois não são considerados como pessoais. \\
A circulação deste capital passa despercebido e usado pelo estado para seus gastos, sendo o cidadão sua fonte, sem nenhum proveito, a não ser que talvez as empresas depois recebem ajudas através desta receita. \\ \\
Em Geral a receita laboral de um cidadão é quase três oitavos $23,75\%+11\%=34,75\%$ depois dos respetivos descontos [1000Eur \textit{vs} 347,5Eur]. \\
\\
Estas contas são feitas sem considerar qualquer subsidio de alimentação. \\
\\
\big[ \; \textcolor{green}{\small link: \quad http://www.seg-social.pt/trabalhadores-por-conta-de-outrem} \; \big]
\subsection{Trabalhadores Independentes}
%%%%%%%%%%%%%%%%%%%%%%%%%%%%%%%%%%%%%%%%%%%%%%%%%%%%%%%%%%%%%%%%%
\qquad Este tipo de contribuinte em princípio pode definir seus descontos numa dada margem, e é aliciante para as empresas este tipo de trabalhador pois não tem qualquer responsabilidade, este acarreta toda a responsabilidade de descontos e despesas, no entanto em principio ira ganhar mais do que o trabalhador por conta de outrem, mas descontando muito menos e prejudicado a longo prazo devido a concorrência, a não ser que desconte a totalidade de $23,75\%+11\%=34,75\%$ e ainda obter um vencimento superior ao seu equivalente de trabalhador por conta de outrem.
Os exemplos são trabalhadores a recibos verdes, subcontratados e a trabalho temporário.
\subsection{Precariedade}
%%%%%%%%%%%%%%%%%%%%%%%%%%%%%%%%%%%%%%%%%%%%%%%%%%%%%%%%%%%%%%%%%
\qquad Nenhum cidadão devia aceitar qualquer trabalho que ganhe menos que \; $ \mbox{\Large $ \frac{635Eur}{0,65}\approx 977Eur $ } $ para se dizer que leva uma vida sustentável, pois o salário mínimo nacional é de 635Eur, e se ficar em \textit{lay off} ou \textit{desempregado}, como demonstrado: \\
\\
\hspace*{.3cm} $635\times(1-0,11)\approx566Eur$,\hspace*{1cm} $635\times(0,3475)\approx220Eur$,\hspace*{1cm} $\frac{635\times14}{12}\times0,65 \approx 482Eur$, \\
\\
estará a trabalhar gratuitamente, só ira receber \textbf{566Eur} com descontos de \textbf{220Eur}, ou seja um escravo do estado. No caso de \textit{lay-off ou desemprego} recebera apenas 482Eur.\\
Em princípio qualquer remuneração será deduzido por: $Vencimento \times (1-0,11) \times (1-0,23) - Combustivel\times 0,61 = Rendimento \, Liquido$, pois tudo também leva \textit{IVA} e a taxa de combustível. \\
\\
\textbf{Exemplo} (\textit{indivíduo com salário mínimo nacional}) \; \textbf{:}\\
\\
1. vencimento = 635Eur e 0Eur gasolina mensal \\
\hspace*{1cm} $635Eur \times (1-0,11) \times (1-0,23) - 0Eur \times 0,61 = 435Eur$ \\
2. vencimento = 635Eur e 80Eur gasolina mensal \\
\hspace*{1cm} $635Eur \times (1-0,11) \times (1-0,23) - 80Eur \times 0,61 = 386Eur$ \\
3. vencimento = 635Eur e 150Eur gasolina mensal \\
\hspace*{1cm} $635Eur \times (1-0,11) \times (1-0,23) - 150Eur \times 0,61 = 343Eur$, \\
\\
mas ainda não acaba aqui a pintura negra, supondo agora que o cidadão não tem caro, ou seja, recebe limpos 435Eur, ainda vai ter que pagar taxa água e saneamento (mínimo 11,3Eur) e taxa de luz (mínimo 8Eur). Fica com 415,7Eur, para piorar vamos supor que tem habitação e têm que pagar IMI (mínimo 11Eur/mês).
Se este exemplo tiver um empréstimo de habitação e ou um veiculo chegamos a conclusão que não pode se alimentar, o que será muito bom para a dieta, e doenças.\\
\\
Concluindo que no estado presente de trabalho só é benéfico se pertencermos aos membros de órgãos estatuários ou político, pois não tem encargos do estado e aufere de regalias e vencimentos mínimo de cinco vezes e até dez vezes superior ao salário mínimo nacional, também existindo casos excecionais de vinte e para cima a mais o salário mínimo nacional. Sendo que esta profissão existe apenas por tráfico de influências e não igualdade ou equidade, muito menos competência, como demonstrado com esta pandemia na qual suas soluções para os problemas são solidariedade.
\subsection{Mudança}
%%%%%%%%%%%%%%%%%%%%%%%%%%%%%%%%%%%%%%%%%%%%%%%%%%%%%%%%%%%%%%%%%
\qquad Já é conhecido que em \textsf{2025}, 75\% da classe trabalhadora vai pertencer a geração \textbf{Y}, e o quadro do futuro de trabalho esta cada vez mais centrado a volta do desenvolvimento tecnológico, as sociedades vão ter que o acompanhar o ritmo de crescimento, e a União Europeia e seus membros reconhecem esta tendência e a necessidade de formação e treino destas competências nos trabalhadores Europeus, sendo o projeto \textit{industria 4.0} uma destas ferramentas.
\begin{figure}[H]
	\centering
	\includegraphics[scale=0.52]{./image/Change/Forces for Change.jpg}
	\caption{Forças para a Mudança \cite{book_7}}
\end{figure}
Empresas de todo tipo e dimensão estão a ser enfrentados com a questão de como podem assegurar o fornecimento de lideres com as competências, habilidades e visão estratégica adequadas para obter o sucesso. Ignorando a velha mentalidade de que certos indivíduos nascem para liderar, muitas empresas acreditam que a liderança pode ser desenvolvida numa forma pro-ativa e de forma sistemática. \cite{book_6}\\
\\
A liderança assume um papel importante na mudança de cultura dentro das organizações e requer estar sempre em constante adaptação ao seu meio ambiente. Como sugere a liderança \textit{VUCA}, que representa Volatilidade, incerteza (uncertainty), complexidade e ambiguidade. Volatile porque não é estático esta e em constante mudança, incerteza na previsão do futuro, complexo com sistemas cada vez mais sofisticados que requer competências adequadas e ambíguo com problemas difícil de identificar, pouca informação de alternativas, sem se saber as consequências.
\begin{figure}[H]
	\centering
	\includegraphics[scale=0.52]{./image/Leadership/Leadership Models.jpg}
	\caption{Modelos da Liderança \cite{book_2}}
\end{figure}
Este trabalho esta focado no futuro do trabalho, gestão de carreira e marketing pessoal, no entanto abordar as matérias da gestão de mudança, a planeada (Modelo Kurt Lewin) e a emergente, os tipos de lideres, seus estilos e abordagens, também os modelos criados (Modelo Blake \& Mouton, Hersey \& Blanchard) são ferramentas úteis para nos orientar nos nossos comportamentos em diferentes contextos e determinar as atitudes a tomar com o nosso grupo ou equipa de forma a poder alcançar os objetivos e uma visão, ou seja garantir a sobrevivência e prosperidade da organização.
\begin{figure}[H]
	\centering
	\includegraphics[scale=0.52]{./image/Change/Three Change Approaches.jpg}
	\caption{Três formas de mudar \cite{book_6}}
\end{figure}
\newpage
\section{O futuro do trabalho}
%%%%%%%%%%%%%%%%%%%%%%%%%%%%%%%%%%%%%%%%%%%%%%%%%%%%%%%%%%%%%%%%%
\qquad Agora com as novas tecnologias tem se aberto várias portas para novas formas de as pessoas poderem ser remuneradas por seus serviços ou bens. Exemplos muito notórios são casos como a UBER, AMAZON, YOUTUBE, LINKEDIN, etc, etc.\\
\\
Esta a fugir para uma forma de trabalhadores independentes, subcontratados e de trabalho temporário, controlado por sistemas tecnológicos administrativos, empresas virtuais, que pode ser formas de exploração e concorrência desleal, quando mal usados, e proporcionam enriquecimento rápido aos que implementem estes sistemas e gerem. \\
\\
Esta ideia já tinha surgido décadas atrás, como uma forma de reduzir custos e responsabilidade do cliente, como a agência de trabalho temporário \textbf{KELLY SERVICES} tinha publicado em 1971 acerca da oferta do tipo de trabalhadores que tinham ao dispor: \cite{book_11} \\
\\
\hspace*{.5cm} - Nunca tiram feriados ou férias\\
\hspace*{.5cm} - Nunca pedem aumentos salariais\\
\hspace*{.5cm} - Nunca custa um cêntimo com folgas de trabalho\\
\hspace*{.5cm} - Nunca fica gripado, problemas de coluna ou dor de dentes\\
\hspace*{.5cm} - Nunca te chateia com situação de desemprego, impostos e segurança social\\
\hspace*{.5cm} - Nunca se cansam de satisfazer\\
\\
Também poderia-se falar do caso da UBER na qual resultou em diversos processos em tribunal.\\
\\ 
Estes acontecimentos servem de exemplo para que a sociedade tenha fortes Leis do trabalho, Direitos humanos e a obrigação de ter líderes conscientes.\\
\\
Agora como foi abordado alguns pontos negativos que se podem encontrar no mundo de trabalho, já se sabe que o mundo foi feito para o ser humano, na qual todos nós somos ao mesmo tempo trabalhadores e clientes, e pretende-se que haja segurança e estabilidade para todos, cada vez mais se valoriza a liberdade, sendo que a esperança no mundo do trabalho seja para exterminar situações de exploração e corrupção.\\
\\
Com a modernização existe uma especial preocupação com a classe trabalhadora com menos formação e a desigualdade na valorização laboral, sendo que quanto maior a procura com menor oferta tem maior o valor.\\
\\
No caso de Portugal o futuro de trabalho esta muito bem estabelecido, esta tabelado, é uma economia por classes onde o governo trabalha de mãos dadas com as organizações, e a classe trabalhadora como é explicito é para trabalhar e manter o sistema a tona da água, seus vencimentos são controlados monitorizados de forma a garantir a competição internacional e a sobrevivência, apenas os quadros superiores e de liderança tem direito a distribuição dos lucros, ou seja os \textit{stakeholders}.
\newpage
Abaixo uma tabela pré-definida de vencimentos para as universidades:\\
\begin{figure}[H]
	\centering
	\includegraphics[scale=0.52]{./image/Salary/Universidade.jpg}
	\caption{Tabela de Vencimentos para 2018 \cite{article_2}}
\end{figure}
Estas publicações são expostas nos Diários da Republica e consequentemente nos Decretos de Lei, claro que quem cria este controle é quem tem os melhores vencimentos, todas as profissões são tabelados com um teto máximo de vencimento, dai que se diz que \textit{"Ninguém fica rico a trabalhar"}, toda a classe trabalhadora seu patronato é o Estado, ou seja, o Governo. As Organizações são Sócios do Governo e pagam uma taxa do uso dos seus cidadãos.\\
\\
No caso dos Engenheiros Eletrotécnicos:\\
Intervalo salarial máximo e mínimo da maioria dos trabalhadores da profissão Engenheiros Eletrotécnicos - a partir de 5,24 € até 16,17 € por hora - Ano 2020. \\
\textcolor{green}{[link: https://meusalario.pt/emprego/portugal-emprego-e-salario/engenheiros-electricos]} \\
\\
Dai o futuro do trabalho é incerto, e tudo depende do governo, isto é da Cultura de Portugal.
\newpage
\section{A gestão de carreira e as competências necessárias num mundo em mudança}
%%%%%%%%%%%%%%%%%%%%%%%%%%%%%%%%%%%%%%%%%%%%%%%%%%%%%%%%%%%%%%%%%
\qquad A Industria tende a ser cada vez mais automatizada, e a mão de obra substituída por maquinas, as empresas estão a ser cada vez mais digital.\\
\\
No futuros os empregos com melhores vencimentos vão ser nas áreas tecnológicas.\\
\input{./group/Competencias}
Acima esta um conjunto de competências que pelos estudos efetuados demonstrou que trabalhadores da industria com maior intensidade digital em média exibem maiores níveis de competência cognitiva e também operacionais do que os trabalhadores nos sectores económicos de menor intensidade digital. Isto claro depende do tipo de trabalhador empregue no sector digital versus o de menor intensidade digital, em que o segundo geralmente são trabalhadores sem qualificações. \cite{article_1}\\
Também é demonstrado que todas as competências tem maior recompensa nas industrias digitalmente intensificadas, particularmente numeração avançada, organização pessoal, tecnologias de informação e comunicação e numeração. \cite{article_1}\\
A recompensa de vencimento pela competência de tecnologias de informação e comunicação é o dobro em relação a competência de numeração, e aonde competências de gestão e comunicação tem recompensa igual as de numeração. \cite{article_1}\\
As competências operacionais ainda estão ao mesmo nível das cognitivas demonstrando forte evidencia da importância de trabalhos orientados a tarefa no mercado de trabalho. \cite{article_1}\\
Compreender quais a competências, tanto como as cognitivas e operacionais, quais são melhor recompensados financeiramente também é importante para responder aos assuntos de  desigualdades e criar empregos e bem estar. A falta da oferta de competências ou conjunto de competências pode facilmente criar desigualdade salarial e desemprego dos trabalhadores sem esses tipos de competências, dai a importância de criar programas de formação para preparar os trabalhadores nessas competências de alta procura dado a aceleração da transformação digital transversal nas ocupações e industria. E quanto mais cedo começa esse treinamento menor os custos de formação das competências necessárias. \cite{article_1}\\
\\
A industria mais tecnologicamente digital paga melhor seus trabalhadores, a formação e treino é necessário para adquirir as competências desejadas para poder prosperar no mercado de trabalho.
%%%%%%%%%%%%%%%%%%%%%%%%%%%%%%%%%%%%%%%%%%%%%%%%%%%%%%%%%%%%%%%%%
%%%%Plano de Careira%%%%
O meu plano de desenvolvimento pessoal, passa por obter mais formação e apreender com pessoas com mais experiência em diversas áreas, que é exatamente o que estou a fazer frequentando o curso de Engenharia Eletrotécnica e de Computadores no \textcolor{gray}{I.S.E.P}.\\
Esta disciplina em particular é uma forma de poder enriquecer minhas competências de Gestão, e as matérias abordadas que compõem o \textcolor{blue}{Comportamento Organizacional}, que são bastantes.
\begin{figure}[H]
	%\centerline
	\begin{minipage}{\linewidth}
		\centering
		\includegraphics[scale=0.3]{"./image/Skills/Managerial Skills for the Global Marketplace.jpg"}
	\end{minipage}
	\caption{Competências de Gestão. \cite{book_6}}
\end{figure}






\begin{comment}
a) Realizar um diagnóstico de competências pessoais;\\
b) Definir objetivos de carreira;\\
c) Definir as competências que considere que no futuro lhe permitirão atingir os referidos objetivos;\\
d) Definir um plano de desenvolvimento para as competências anteriormente selecionadas.\\
Eliminate waste, reduce errors and improve.
\end{comment}
\newpage
\section{Conclusão}
%%%%%%%%%%%%%%%%%%%%%%%%%%%%%%%%%%%%%%%%%%%%%%%%%%%%%%%%%%%%%%%%%
\qquad O futuro do trabalho passa por adquirir novas competências, e conseguir adaptação as novas tendências, a procura de oportunidades e valorização pessoal uma mais valia. Já foi demonstrado que é importante desenvolvermos tanto as nossas metodologias de trabalho como a experiência para estarmos preparados para enfrentar os desafios que possam surgir. \\
\\
Ninguém sabe o futuro, muito menos o imprevistos, cabe as novas gerações decidir, a liberdade acho que é algo que todos desejam, a estabilidade e segurança, mas não só depende de nós, o planeta terra, a galáxia e o universo tem palavra soberana.\\
\\
Considero a humildade e gratidão um atributo fundamental, saber que estamos sujeitos a forças maiores e respeitar essas fronteiras, historicamente isso foi comprovado vezes sem conta, e hoje é outra prova disso, como a pandemia.\\
\\
Gerações após gerações existe uma concentração tremenda muito focada com uma visão míope, um convite para o desastre. Sabe-se lá se temos cura. No entanto temos que ter esperança que haja visão e iluminação.
Esta conversa até parece religiosa, mas nada disso.\\
\\
Ainda muitas conclusões pode-se tirar acerca do futuro do trabalho e marketing pessoal, que não entra nos parâmetros deste relatório e discussão, o desenvolvimento é uma forma de enriquecimento mais rápido e eficaz, com contornos sociais complicados, da a entender que vai haver um excesso de produção sem clientes á vista, um problema de reciclagem e sobrevivência, talvez devia haver uma preocupação na  regulamentação.\\
\\
O ensino talvez esta a ficar desadequado para nossos tempos, não incluindo as ferramentas necessárias na formação e treino, o tempo de retenção dos estudantes excessivo, que provoca colisões entre gerações, especialmente num país pequeno com uma cultura estatística, sendo quase impossível obter os resultados esperados caindo na decadência social e económica no seu geral, etc, etc, etc.
\begin{comment}
Não conheço nenhuma geração que não tenha passado por dificuldades tremendas, episódios horrendos. Parece existe mudança mas fica tudo igual, o sistema já esta implementado. \\
Para sobreviver temos de nos adaptar e competir, mas a maquina é a mesma. A gerência parece estática e firme é uma herança perpetua de desigualdade. \\
\end{comment}
%%%%%%%%%%%%%%%%%%%%%%%%%%%%%%%%%%%%%%%%%%%%%%%%%%%%%%%%%%%%%%%%%
\newpage
%%%%%%%%%%%%%%%%%%%%%%%%%%
%\input{./input/EQUACAO}
%\input{./input/DEFINICAO}
%%%%%%%%%%%%%%%%%%%%%%%%%%
%Figuras Bibliografia Index
\listoffigures
\cite{*}
\bibliography{./bibliography/Bibliography}
%\printindex
\newpage
\footnote{Apontamento}
\end{document}
%%%%%%%%%%%%%%%%%%%%%%%%%%%%%%%%%%%%%%%%%%%%%%%%%%%%%%%%%%%%%%%%%
\begin{comment}

\end{comment}
%%%%%%%%%%%%%%%%%%%%%%%%%%%%%%%%%%%%%%%%%%%%%%%%%%%%%%%%%%%%%%%%%