%\documentclass[titlepage, a4paper, 12pt, reqno, openany]{beamer}
%\documentclass[12pt]{beamer}
\documentclass[aspectratio=169]{beamer}
%%%%%%%%%%%%%%%%%%%%%%%%%%%%%%%%%%%%%%%%%%%%%%%%%%%%%%%%%%
%\usepackage{titlepic}
%%%%%%%%%%%%%%%%%%%%%%%%%%%%%%%%%%%%%%%%%%%%%%%%%%%%%%%%%%
\usepackage[utf8]{inputenc}
\usepackage[T1]{fontenc}
%%%%%%%%%%%%%%%%%%%%%%%%%%%%%%%%%%%%%%%%%%%%%%%%%%%%%%%%%%
\usepackage{graphics}
\usepackage{graphicx}
\usepackage[graphicx]{realboxes}
\usepackage[font=small, labelfont=bf]{caption}
\usepackage{lmodern}
\usepackage{hyphenat}
\usepackage{color,colortbl}
%%%%%%%%%%%%%FIXED ORDER%%%%%%%%%FIXED ORDER%%%%%%%%%%%%%%
%\usepackage[top=2.5cm,left=3cm,right=2.5cm,bottom=2.5cm]{geometry}
%%%%%%%%%%%%%%%%%%%%%
\usepackage[portuguese]{babel}
\usepackage{scrbase}
\usepackage{babelbib}
%%APPENDIX SETTINGS%%
\usepackage[title, toc, titletoc, page]{appendix}
\usepackage[toc, page]{appendix}
%%%%%%%%%%%%%%%%%%%%%
\usepackage{subcaption}
\usepackage{verbatim}
\usepackage{pgfgantt}
\usepackage{url}
\usepackage{listings}
\usepackage[calc, en-US]{datetime2}
\usepackage{etoolbox}
\usepackage{xparse}
%%%%%%%%%%%%%%%%%%%%%%%%%%%%%%%%%%%%%%%%%%%%%%%%%%%%%%%%%%
%\usepackage[nottoc]{tocbibind}
%\usepackage[shortcuts, nopostdot, acronym, toc, nogroupskip, nonumberlist, automake]{glossaries}
%\usepackage[usenames, dvipsnames, svgnames, table]{xcolor}
%\usepackage[hidelinks]{hyperref}
%%%%%%%%%%%%%%%%%%%%%%%%%%%%%%%%%%%%%%%%%%%%%%%%%%%%%%%%%%
\usepackage{multicol}
\usepackage{makecell}
\usepackage{array}
\usepackage{tabularx}
\usepackage[export]{adjustbox}
\usepackage{eurosym}
\usepackage{float}
%%%%%%%%%%%%%%%%%%%%%%%%%%%%%%%%%%%%%%%%%%%%%%%%%%%%%%%%%%
\usepackage{amsmath}
\usepackage{amsfonts}
\usepackage{amssymb}
\usepackage{mathrsfs}
\usepackage{paralist}
\usepackage{enumerate} %%conflict never put enumitem with enumerate
%\usepackage{enumitem} %%conflict never put enumitem with enumerate
\usepackage{multirow}
\usepackage{lscape}
%%%%%%%%%%%%%%%%%%%%%%%%%%%%%%%%%%%%%%%%%%%%%%%%%%%%%%%%%%
\usepackage{scrhack}
\usepackage{longtable}
\usepackage{booktabs}
\usepackage[autostyle=true]{csquotes}
\usepackage{calc}
\usepackage{siunitx}
\usepackage{setspace}
%%%%%%%%%%%%%%%%%%%%%%%%%%%%%%%%%%%%%%%%%%%%%%%%%%%%%%%%%%
\usepackage{lipsum}
%%%%%%%%%%%%%%%%%%%%%%%%%%%%%%%%%%%%%%%%%%%%%%%%%%%%%%%%%%
\usepackage{moreverb}
\usepackage{rotating}
%%%%%%%%%%%%%%%%%%%%%%%%%%%%%%%%%%%%%%%%%%%%%%%%%%%%%%%%%%
\usepackage{romannum}
%\usepackage{tikz}
%\usepackage{circuitikz}
%\usetikzlibrary{matrix, shapes.geometric, arrows, trees, positioning, calc}
%%%%%%%%%%%%%%%%%%%%%%%%%%%%%%%%%%%%%%%%%%%%%%%%%%%%%%%%%%%%%%%%
\usetheme{Frankfurt}
%\makeindex
%%%%%%%%%%%%%%%%%%%%%%%%%%%%%%%%%%%%%%%%%%%%%%%%%%%%%%%%%%%%%%%%
%\bibliographystyle{plain}
%\bibliographystyle{ieeetr}
%\selectbiblanguage{portuguese}
%\setbtxfallbacklanguage{english}
%%%%%%%%%%%%%%%%%%%%%%%%%%%%%%%%%%%%%%%%%%%%%%%%%%%%%%%%%%%%%%%%
\title{Comportamento Organizacional}

\author{Sérgio Santos}

\date{\today}
\maketitle
%%%%%%%%%%%%%%%%%%%%%%%%%%%%%%%%%%%%%%%%%%%%%%%%%%%%%%%%%%%%%%%%%%%%%%%%%%%%%%%%%%%%%%%%%%%%%%%%%%%%%%%%%%%%%%%
%\begin{frame}{Nome da Organização}
%\tableofcontents
%\end{frame}
%%%%%%%%%%%%%%%%%%%%%%%%%%%%%%%%%%%%%%%%%%%%%%%%%%%%%%%%%%%%%%%%%%%%%%%%%%%%%%%%%%%%%%%%%%%%%%%%%%%%%%%%%%%%%%%
\AtBeginSection[]{
\begin{frame}
\frametitle{ROQ}
\tableofcontents[currentsection]
\end{frame}}
%%%%%%%%%%%%%%%%%%%%%%%%%%%%%%%%%%%%%%%%%%%%%%%%%%%%%%%%%%%%%%%%%%%%%%%%%%%%%%%%%%%%%%%%%%%%%%%%%%%%%%%%%%%%%%%
%%%%%%%%%%%%%%%%%%%%%%%%%%%%%%%%%%%%%%%%%%%%%%%%%%%%%%%%%%%%%%%%%%%%%%%%%%%%%%%%%%%%%%%%%%%%%%%%%%%%%%%%%%%%%%%
\section{Introduction}
%%%%%%%%%%%%%%%%%%%%%%%%%%%%%%%%%%%%%%%%%%%%%%%%%%%%%%%%%%%%%%%%%%%%%%%%%%%%%%%%%%%%%%%%%%%%%%%%%%%%%%%%%%%%%%%
\begin{frame}
\frametitle{\textcolor{green}{S. Roque}}
\begin{center}
\end{center}
\begin{center}
\begin{tikzpicture}
\node[anchor=south west,inner sep=0] (image) at (0,0) {\includegraphics[width=1\textheight]{./image/ROQ/ROQ_Pavilhoes.jpg}};
%\node[align=left,red,font={\normalsize\bfseries}] at (image.center) {Do NOT eat\\these!};
\end{tikzpicture}
\end{center}
\end{frame}




\begin{frame}
\frametitle{1979-1983}
Sr. Manuel Sá se estabeleceu por conta própria, ocupando a garagem de uma habitação de um familiar, no Lugar de S.Roque (Freguesia de Riba de Ave, Concelho de Vila Nova de Famalicão), deu início a uma empresa vocacionada para a prestação de todo o tipo de serviços relativos à serralharia mecânica. Centrou desde logo a sua atividade nas empresas têxteis da região, onde a manutenção dos equipamentos, quase na sua totalidade importados, proporcionavam um mercado de trabalho com inúmeras oportunidades.
\end{frame}

\begin{frame}
\frametitle{1983-1984}
Entrada do Sr. Joaquim Sá é constituída uma sociedade por quotas, com capital social de 1 milhão de escudos e dois postos de trabalho, sendo atribuída a denominação de Serralharia Mecânica S.Roque, Lda.
\end{frame}

\begin{frame}
\frametitle{1984-2001}
A empresa criou a sua primeira máquina automática de estampar artigos têxteis com formato circular. Desde então dedicou-se à criação do fabrico destas máquinas, marcando definitivamente a evolução da empresa. Na mesma época também iniciou o fabrico de Estufas de Termofixação e de Rámulas. O seu crescimento levou ano após ano a um aumento progressivo do número de colaboradores, bem como à necessidade de a empresa procurar um espaço mais adequado à nova realidade.\\
No início da década de noventa a S.Roque muda de instalações, para Oliveira S. Mateus no mesmo concelho. Não descurando a evolução dos seus equipamentos para a estamparia, cria uma nova linha de produtos, máquinas para automatização do setor da embalagem. Simultaneamente, na procura de novos mercados, a empresa inicia a promoção dos seus equipamentos através da sua participação em exposições.
\end{frame}

\begin{frame}
\frametitle{2001-2004}
Surge a divisão do laser, com a designação de "Roqlaser" cujo objectivo é o fabrico de peças metálicas, através da utilização de tecnologia de vanguarda na área do corte a laser, quinagem e soldadura.
\end{frame}

\begin{frame}
\frametitle{2004 até Hoje}
S.Roque tem vindo a ter uma evolução acentuada a nível financeiro, tecnológico e humano, para puder atuar num mercado cada vez mais competitivo. Investiu na construção de novos pavilhões, aquisição de novas máquinas e contratação de mão-de-obra profissional, de modo a puder continuar a acompanhar um mercado cada vez mais exigente.
\end{frame}

\begin{frame}
\frametitle{2015}
{\Huge 2015}\\
S.roque transformou-se em ROQ. Atendendo às necessidades do seculo XXI reconhecemos a necessidade de criar uma marca verdadeiramente global que consiga de uma forma eficaz transmitir mais de 30 anos de história, inovação, internacionalização e conhecimento.
\end{frame}

%\end{itemize}
%\end{frame}
%%%%%%%%%%%%%%%%%%%%%%%%%%%%%%%%%%%%%%%%%%%%%%%%%%%%%%%%%%%%%%%%%%%%%%%%%%%%%%%%%%%%%%%%%%%%%%%%%%%%%%%%%%%%%%%
\section{Organização}
%%%%%%%%%%%%%%%%%%%%%%%%%%%%%%%%%%%%%%%%%%%%%%%%%%%%%%%%%%%%%%%%%%%%%%%%%%%%%%%%%%%%%%%%%%%%%%%%%%%%%%%%%%%%%%%
\begin{frame}
\frametitle{dah dah dah}

asadfsdgfhfhfd

\begin{theorem}{lei de ohm}
this is the theorem block
\end{theorem}
\begin{definition}
definition
\end{definition}
\begin{example}
example
\end{example}

\end{frame}
%%%%%%%%%%%%%%%%%%%%%%%%%%%%%%%%%%%%%%%%%%%%%%%%%%%%%%%%%%%%%%%%%%%%%%%%%%%%%%%%%%%%%%%%%%%%%%%%%%%%%%%%%%%%%%%
\begin{frame}

\begin{block}{o que é}
da
\end{block}
\begin{alertblock}{o que é}
da
\end{alertblock}
\begin{exampleblock}{o que é}
da
\end{exampleblock}

\end{frame}
%%%%%%%%%%%%%%%%%%%%%%%%%%%%%%%%%%%%%%%%%%%%%%%%%%%%%%%%%%%%%%%%%%%%%%%%%%%%%%%%%%%%%%%%%%%%%%%%%%%%%%%%%%%%%%%
\section{Modelo Ogbonna \& Haris}
%%%%%%%%%%%%%%%%%%%%%%%%%%%%%%%%%%%%%%%%%%%%%%%%%%%%%%%%%%%%%%%%%%%%%%%%%%%%%%%%%%%%%%%%%%%%%%%%%%%%%%%%%%%%%%%
\begin{frame}
\frametitle{asdfasdf}

asdfasdf

\end{frame}
%%%%%%%%%%%%%%%%%%%%%%%%%%%%%%%%%%%%%%%%%%%%%%%%%%%%%%%%%%%%%%%%%%%%%%%%%%%%%%%%%%%%%%%%%%%%%%%%%%%%%%%%%%%%%%%
\subsection{shit}
something
%%%%%%%%%%%%%%%%%%%%%%%%%%%%%%%%%%%%%%%%%%%%%%%%%%%%%%%%%%%%%%%%%%%%%%%%%%%%%%%%%%%%%%%%%%%%%%%%%%%%%%%%%%%%%%%
\begin{frame}
\frametitle{Images in Beamer}

\includegraphics[scale=.25]{"./image/OB/Ogbonna & Harris.jpg"}

\end{frame}
%%%%%%%%%%%%%%%%%%%%%%%%%%%%%%%%%%%%%%%%%%%%%%%%%%%%%%%%%%%%%%%%%%%%%%%%%%%%%%%%%%%%%%%%%%%%%%%%%%%%%%%%%%%%%%%
\begin{frame}
\frametitle{columns}

\begin{columns}
\column{.5\textwidth}

\includegraphics[scale=.1]{"./image/OB/Ogbonna & Harris.jpg"}

\column{.5\textwidth}

sdafdgfshghserhrehsregsegsfdg

\end{columns}

\end{frame}
%%%%%%%%%%%%%%%%%%%%%%%%%%%%%%%%%%%%%%%%%%%%%%%%%%%%%%%%%%%%%%%%%%%%%%%%%%%%%%%%%%%%%%%%%%%%%%%%%%%%%%%%%%%%%%%
\section{Cultura Organizacional}
%%%%%%%%%%%%%%%%%%%%%%%%%%%%%%%%%%%%%%%%%%%%%%%%%%%%%%%%%%%%%%%%%%%%%%%%%%%%%%%%%%%%%%%%%%%%%%%%%%%%%%%%%%%%%%%
\begin{frame}
uygyujyujuhg
trdhrdhf
dtrdhtrdg


\end{frame}

\section{Conclusões}
%%%%%%%%%%%%%%%%%%%%%%%%%%%%%%%%%%%%%%%%%%%%%%%%%%%%%%%%%%%%%%%%%%%%%%%%%%%%%%%%%%%%%%%%%%%%%%%%%%%%%%%%%%%%%%%
\begin{frame}
uygyujyujuhg
trdhrdhf
dtrdhtrdg


\end{frame}

%%%%%%%%%%%%%%%%%%%%%%%%%%%%%%%%%%%%%%%%%%%%%%%%%%%%%%%%%%%%%%%%%%%%%%%%%%%%%%%%%%%%%%%%%%%%%%%%%%%%%%%%%%%%%%%
%%%%%%%%%%%%%%%%%%%%%%%%%%%%%%%%%%%%%%%%%%%%%%%%%%%%%%%%%%%%%%%%%%%%%%%%%%%%%%%%%%%%%%%%%%%%%%%%%%%%%%%%%%%%%%%
\end{document}
%%%%%%%%%%%%%%%%%%%%%%%%%%%%%%%%%%%%%%%%%%%%%%%%%%%%%%%%%%%%%%%%%%%%%%%%%%%%%%%%%%%%%%%%%%%%%%%%%%%%%%%%%%%%%%%