\input{./input/PREAMBLEREPORT_1}
%
\begin{minipage}{\linewidth}

\title{Comportamento Organizacional}
\author{
\emph{S\'{e}rgio Santos},\;$N^o$:\; 1020881 \\
\emph{Nome 2},\;$N^o$:\; 2000000\\
\emph{Nome 3},\;$N^o$:\; 3000000\\
%\emph{Nome 4},\;$N^o$:\; 4000000\\
%\emph{Nome 5},\;$N^o$:\; 5000000\\
}
\date{\today}
%\titlepic{\includegraphics[scale=0.50]{./image/ROQ/ROQ.jpg}}

\begin{titlepage}
\includegraphics[scale=0.60]{./image/capa/ISEP_marca_cor_grande.png}
\maketitle
\vspace{8cm}
\begin{flushleft}
\includegraphics[scale=0.50]{./image/ROQ/ROQ.jpg}
\end{flushleft}
\end{titlepage}

\end{minipage}

%
\tableofcontents
%
%\appendix
%
\pagestyle{plain} %plain headings empty
%\setcounter{chapter}{0}
%\numberwithin{page}{section}
%%%%%%%%%%%%%%%%%%%%%%%%%%%%%%%%%%%%%%%%%%%%%%%%%%%%%%%%%%%%%%%%%%%%%%%%%%%%%%%%%%%%%%%%%%%%%%%%%%%%%%%%%%
\newpage
\label{Resumo}
\begin{abstract} 
Este trabalho consiste na análise de uma organização, quanto ao Comportamento, Cultura e Liderança.\\

A Cultura Organizacional é fundamental para as organizações poder evoluir e atingir seus objectivos com sucesso. O estudo das culturas presentes nas organizações e formas de à moldar para melhor servir a sociedade e mercado sera abordado neste relatório.
\end{abstract}
%%%%%%%%%%%%%%%%%%%%%%%%%%%%%%%%%%%%%%%%%%%%%%%%%%%%%%%%%%%%%%%%%%%%%%%%%%%%%%%%%%%%%%%%%%%%%%%%%%%%%%%%%%%
\newpage
\section{Introdução}
Durante muito tempo tem havido estudos para descobrir a formula mágica que leva as organizações ter sucesso, existe muitas abordagens com diferentes perspectivas, algumas convergem e outras divergem, e assim foram criados novos conceitos de forma a encapsular estilos e tipos de forma a poder se identicar quais são os mais propicios a ter uma maior elevada taxa de sucesso, conceitos tais como comportamento organizacional, cultura organizacional, performance organizacional, etc.\\

Tem se aplicado o método scientifico para se poder medir e quantificar suas influências, assim obtemos incertezas quantificaveis dando uma ideia qual as mais provaveis levantado novas questões e duvidas das suas interações e até posiveis novas variaveis.\\

Temos ao nosso dispor um extenso leque de trabalhos nesta matéria, até considerado exagerado, que pelo menos nos indica um caminho que vale apena percorrer e explorar, graças aos individuos que se dedicaram a esta desciplina.\\


\section{Organização}

Uma organização é um grupo estruturado de pessoas, um grupo onde cada pessoa é responsável por tarefas bem definidas e onde existe um sistema de articulação entre elas, que desenvolve um conjunto de actividades visando a definição e prossecução de objectivos comums (de forma continuada no tempo).\\

O objectvo das organizações é para criar valor para os seus clientes/utentes, para os detentores do seu capital, para os seus colaboradores, para os seus fornecedores e para a sociedade em geral.\\

A Empressa na qual vai ser analisada neste trbalaho vai ser ..........................

As empresas privadas são obrigadas a oferecer valor aos seus (potenciais) clientes, aos detentores do seu capital, aos trabalhadores e aos fornecedores.\\
Nas empresas privadas, existe um forte estimulo do mercado (e da concorência) para que ela estabeleça objectivos bem definidos e socialmente apetecidos para poder sobreviver e prosperar.\\

Nas empresas publicas, o estimulo é sentido em menor grau ou não existente, consoante o tipo e caracteristicas concretas da organização.\\

Nas organizações sem fim lucrativos e/ou dependentes do estado, existe outro tipo de estimulo os interesses dos governantes, quando o estado é parceiro ou responsavel pela organização, os interesses de grupos de pressão da sociedade, etc.\\

Nas organizações sem fins lucrativos os objectivos estão em permanente discusão e/ou a ser alterados, resultando uma maior indefinição sobre as actividades a desenvolver por responsaveis e por colaboradores.\\

Teoria de Geer Hofteed.\\

Best practices can come from national, say the American National Standards Institute (ANSI) or the Canadian Standards Association (CSA), or international, say ISO or Institute of Electrical and Electronics Engineers (IEEE), standards organizations, professional associa-tions, or consulting firms.\\

Eliminar desperdicio e resolução de problemas.\\

Confiança na liderança e operadores.\\

Both organizational culture and a leader’s behaviors have been identified as critical determinants of an organization’s effectiveness\\

Leaders have long been viewed as a primary influence on the creation of organizational culture (e.g. Bennis and Nanus, 1985; Schein, 1983). According to Schein (1985), the “only thing of real importance that leaders do is to create and manage culture”\\


% A vida ensina e se não aprendemos ela ensiste e persiste até morrermos.\\

%A liberdade é medida pela quantidade de ethica e moralidade presente na sociedade.\\

%Respect is always earned never a given.\\


No organization today exists in a stable environment. Current scholars, especially the
proponents of complexity theories, consider that all organizations are under the influence
of multiple changes (Brown and Eisenhardt, 1997; Burnes, 2004a; Stacey et al., 2002;
Styhre, 2002; Tetenbaum, 1998). According to these scholars, change is inherent in
human action and therefore necessarily occurs in any context of human social interactions
(Ford and Ford, 1995). As organizations are sites of continuously evolving human action,
they are in a continuous state of change and, in order to survive, must develop the ability
to continuously change themselves (Burnes, 2004b; Tsoukas and Chia, 2002).\\

Constante mudança de adaptaçao ao meio ambiente.\\


Organizational culture determines how individuals behave, what people pay attention to,
how they respond to different situations, and how they socialize with new members and
exclude those who do not fit in (Spataro, 2005) \\

acknowledge that even in US and European companies, success rates are
not spectacular regarding efforts to change vision, values, and culture or business systems
and processes (Beer and Nohria, 2000; Beer et al., 1990; Carr et al., 1996).\\

Characteristica de bons Objectivos\\
- Claros\\
- Concisos\\
- Calendarizados\\
- Atingiveis\\


Tipos de organizações\
- Organização privadas com fins lucrativos\\
- Organização privadas sem fins lucrativos\\
- Organização publicas com fins lucrativos\\
- Organizações publicas sem fins lucrativos\\


Outsourcing\\



\section{Estrutura da Organização}
Quais as culturas possiveis das organizações na qual podem derivar do contexto onde se encontra, e quais não são possiveis.\\

Concentrar em Portugal, prós e contras.\\

tipos de hierarquias\\
tipos de departamentalizações\\
organização por processo\\

A divisão do trabalho, permitiu a redução do tempo de aprendizagem, isto é, cada um tem as suas funções, aumentando a produtividade. Cada um executa uma parte das tarefas necessarias a fabricação.\\

\newpage
Gestão:\\ \\
\begin{minipage}{20cm}
\begin{minipage}{5cm}
Instrumentos
\begin{enumerate}
\item Planear
\item Organizar
\item Controlar\\ \\
\end{enumerate}
\end{minipage}
\begin{minipage}{5cm}
Funções
\begin{enumerate}
\item Liderança
\item Comuniacção
\item motivação
\item Tomada de decisão
\end{enumerate}
\end{minipage}
\end{minipage}

Cadeia de valor\\
-Actividades principais\\
-Actividades de supporte\\

Cadeia de valor da organização é a sequencia de actividades e fluxos de informação que uma organização e os seus fornecedores devem desenvolver para desenhar, produzir, oferecer, entregar e suportar os seus produtos, estas são as actividades principais.\\

As actividades de suporte são as que apoiam um bom desempenho na realização das actividades principais.\\
- actividade administrativa e financeira\\
- actividade da gestão do pessoal\\
- actividade juridica\\
- planeamento, controlo e gestão\\
- gestão de sistemas e tecnologia\\

A actividade de suporte não contribuem directamente para a criação do valor.\\


Actividades de suporte e principal.\\
funções da Gestão sã Instrumental, Comportamental e Estrutural.\\

Cumprir os objectivos é ser eficaz.\\

Para gerir a produção (planear, organizar, dirigir e controlar), há que recolher um elevado volume de informação de controlo, sendo frequentemente necessário refazer o planeamento.\\

- Implementação por projecto\\
- Implementação por processo\\
- Implementação por células\\
- Implementação por cadeia ou em linha\\
- Implementação por produto\\

Na implementação por célula de fabrico procura agrupar os produtos segundo a semelhança das suas rotinas operatórias.\\
Na implementação por processo, é possivel cada serie (ou lote) ser processado integralmente num dado centro, antes de avançar para o centro onde irá sofrer a operação de transformação seguinte.\\
A análise ABC pode ser utilizada para averiguar quais as principais encomendas responsáveis pela sobrecarga de um dado centro de trabalho.\\

Organizar é estipular quem faz o quê, atribui-se os recursos necessários para o fazer, criar um sistema de informação para verificar execução.\\



\section{Cultura Organizacional}
Planeamento\\

Planear é estabelecer os objectivos a atingir e o percurso de acções.\\

\subsection{Missão}
A missão de uma organização consiste na sua razão de existir, actual e futura.\\
A formulação, avaliação e selecção de estratégias e o desenvolvimento dos planos mais detalhados para as pôr em práctica são feitos após a definição da missão e da análise do meio ambiente da organização.\\



\subsection{Visão}



\subsection{Valores}



\subsection{Lema}


Ferramentas para avaliar o cumprimento dos objectivos\\
- benchmarking\\
- scorecard management\\
- Banco de Portugal\\

Metodo de demonstar o desempenho de uma organizações atraves da eficacia, eficiencia e seu rendimento.\\

Eficacia avalia em que medida os objectivos estão alinhados com a necessidades sociais que ela se propõe a satisfazer, ou seja, em que medida os seus objectivos são a tal adequados.\\

Eficiencia avalia a economia de recursos utilizados para realizar os seus objectivos, requer uma boa estruturação dos processos seguidos nas actividades, o que leva tempo e custa dinheiro.\\

Missão - SWOT Meio Ambiente (transacional e contextual(PEST)) - Objectivos - Implementação.\\

O sucesso das empresas está correlacionada positivamente com o seu planeamento.\\

SWOT\\

No planeamento estratégico, o dianostico do meio ambiente pode ser feito recorrendo a uma análise SWOT.\\

Na análise do meio ambiente transacional, análisa-se o comportamento previsional das entidades com quem a organização interage.\\

Controlo.\\
O controlo pode ser encarado como um processo de aprendizagem.\\
O controlo deve servir, acima de tudo, para ajudar a garantir que os objectivos estabelecidos são atingidos.\\
Se a informação recolhida e os resultados apurados no processo de controlo não conduzem a acções de correcção quando necessário, este será não só inútil, mas até prejudicial.\\
O recurso aos sistemas de informação permite, em geral, simplificar os procedimentos de controlo.\\

Controlar é os procedimentos de verificar sua execução, estar atento a imprevistos e pronto a corecções recorrendo a re-organização e/ou novo planeamento, também pode-se optar por não fazer nada.\\




\section{Estudo da Cultura Organizacional segundo o Modelo Ogbonna \& Harris}


define first the terms.\\



\begin{figure}[H]
\centering
\includegraphics[scale=.3]{"./image/OB/Ogbonna & Harris.jpg"}\\
\caption{Ogbonna \& Harris}
\label{grafico 1}
\end{figure}\par



Motivação teorias Moslow, Hersberg, Victor Vroom.\\
Se as condições e recompensas oferecidas aos funcionários não lhes permitirem satisfazerem algumas das suas necessidades, mais facilmente abandonam a equipa ou organizaçãoa que pertencem.\\
Motivação é o conjunto de factores que provocam, canalizam e sustentam o comportamento das pessoas.\\

Um gestor interessado em atingir um bom desempenho estabelece objectivos atingiveis e bem defenidos.\\
Auto-confiança no desenvolvimento do trabalho pode deminuir a sua motivação.\\
Motivar é criar condições necessarias para que as pessoas se empenham na prosecução dos objectivos da organização.\\

O impacto da comunicação no desempenho da organização é muito elevado.\\
Os Gestores tem de ser coerentes e alinhados com o que transmittem de forma a criar uma estrutura de confiança.\\

SWOT\\
PEST\\

Um lider de uma organização é aquele que detém capacidades de influenciar os colaboradores.\\

gestão das actividades é o exercicio do poder de um gestor.\\

poder de premiar e punir é suficiente para gerir as actividades do dia a dia.\\

poder informacional.\\

Um sistema de avaliação do desempenho de uma empresa é uma valia porque é uma boa oportunidade para analisar o grau de cumprimento dos objectivos acordados.\\

A medida que as organizações se achatam, os gestores têm de aprender a permitir que os seus colaboradores tomem decisões e tenham informação sobre questões mais sensiveis.\\

Para um engenheiro é muito útil conhecr os aspectos essenciais da legislação laboral.\\

A gestão das pessoas é cada vez mais importante porque são as pessoas que têm o conhecimento e só as pessoas o podem partilhar e aplicar.\\

Numa organização, apoiar e compensar as pessoas é fundamental, mas também é indespensávelfalar com elas sobre os erros que cometem no sentido de serem corrigidas e evitadas no futuro.\\

Uma organização tem maior probabilidade de ter sucesso se gerir as pessoas de modo a que estas ao contribuirem para o sucesso da organização tenham também sucesso elas próprias.\\

A qualidade é a totalidade das caracteristicas de um produto ou serviço, que determinam a sua aptidão para satisfazer determinadas necessidades.\\

Garantia da qualidade tem como objectivo primeiro, o controlo do processo, ou seja, a minimização ou mesmo eliminação dos erros na produção.\\

A garatia da qualidade concentra-se no controlo do processo produtivo e controlo do produto.\\

Os custos relacionados com a insatisfação dos clientes são considerados custos de não qualidade.\\

O diagrama de Pareto permite identificar rapidamente as causas vitais e as triviais de um dado problema.\\

As cartas de controlo destinam-se a detectar as variações resultantes da alteração, frequentemente de natureza aleatória e acidental, de algum dos parametros de processo de fabrico (ditas causas especiais).\\

As sete ferramentas classicas da qualidade\\
- Fluxograma\\
- Registo e análise de dados\\
- Diagrama de causa - efeito (espinha de peixe 4M)\\
- Diagrama de Pareto\\
- Histogramas\\
- Diagramas de dispersão\\
- Cartas de Controlo\\

tipos de lideres\\
- Autocratico\\
- Paricipativo \\
- Democratico \\
- Deixa andar\\

estilo de lider\\
- Orientado as pessoas
- Orientado as tarefas\\





\newpage
\section{Conclusões}
The relationship between culture and leadership appears to be reciprocal—top leaders
create and maintain an organizational culture, which in turn influences the values, atti-
tudes, and behaviors of middle and entry-level leaders. Although leader–culture fit has not
been specifically studied in the published literature, we believe that there is value in exam-
ining the match between a leader’s behaviors and the culture in which they work. While
research hasn’t examined this at the level we discuss, current research does suggest that fit
is important at the national level and at the leader–follower level. Expansion of this
research will help determine what aspects of leader–culture fit are determinants of leader
and organizational effectiveness.
Although there are a variety of approaches that researchers can take to examining
leader–culture fit, we offer the following recommendations. First, although studies of
perceived fit are of limited value, the ease of collecting this data should motivate researchers
to start thinking about adding questions concerning leader–culture fit. Given the lack of
published findings, this research can begin shaping our knowledge about this phenomenon.
Second, while studies of subjective fit will be more important, researchers should utilize
360-degree measurement systems in order to also obtain the most objective fit indices
possible. This practice will likely tell us more about the impact of fit than just examining
leaders’ self-reports. Third, it is important to measure culture at the aggregate level in
order to ensure that the actual values of the organization are being captured, not just the
leader’s values. Although these recommendations may be difficult to achieve in practice,
they offer the best hope of leveraging leader–culture fit for the future.\\

A actividade comportamental consiste em lidar com pessoas, comunicar instruções e receber feedback, propiciar a comunicaçãoentre os terceiros, motivar, tomar decisões e criar condições para que os colaboradores também o possam fazer, assegurar a liderança para comprir a execução.\\

\newpage
%%%%%%%%%%%%%%%%%%%%%%%%%%%%%%%%%%%%%%%%%%%%%%%%%%%%%%%%%%%%%%%%%%%%%%%%%%%%%%%%%%%%%
%\input{./input/EQUACAO}
%\input{./input/DEFINICAO}
%\newpage
%\input{Equacoes}
%%%%%%%%%%%%%%%%%%%%%%%%%%%%%%%%%%%%%%%%%%%%%%%%%%%%%%%%%%%%%%%%%%%%%%%%%%%%%%%%%%%%%%
%Figuras Bibliografia Index
\listoffigures
\cite{*}
\bibliography{./bibliography/Bibliography}
%outro metodo mas manual\input{Bibliografia}
%\printindex
\newpage
\footnote{Apontamento}
\end{document}
%%%%%%%%%%%%%%%%%%%%%%%%%%%%%%%%%%%%%%%%%%%%%%%%%%%%%%%%%%%%%%%%%%%%%%%%%%%%%%%%%%%%%%
