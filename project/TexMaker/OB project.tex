\input{./input/PREAMBLE}
%
\include{./include/CAPA}
%
\tableofcontents
%
%\appendix
%
\pagestyle{plain} %plain headings empty
%\setcounter{chapter}{0}
%\numberwithin{page}{section}
%%%%%%%%%%%%%%%%%%%%%%%%%%%%%%%%%%%%%%%%%%%%%%%%%%%%%%%%%%%%%%%%%%%%%%%%%%%%%%%%%%%%%%%%%%%%%%%%%%%%%%%%%%
\newpage
\label{Resumo}
\begin{abstract}
Este trabalho consiste no estudo do Comportamento Organizacional para desenvolvermos competências de liderança nas organizações para seu desenvolvimento e prosperidade.\\

A Cultura Organizacional é fundamental para as organizações poder innovar, evoluir e atingir seus objectivos com sucesso. O estudo da cultura presente nas organizações e formas de à moldar para melhor servir a communidade e mercado sera abordado neste relatório.
\end{abstract}
%%%%%%%%%%%%%%%%%%%%%%%%%%%%%%%%%%%%%%%%%%%%%%%%%%%%%%%%%%%%%%%%%%%%%%%%%%%%%%%%%%%%%%%%%%%%%%%%%%%%%%%%%%%
\newpage
\section{Introdução}

RASCUNHOS \\ \\

Começar pela teoria de Geer Hfteed porque é o estudo da nação, depois ir para as teorias das organizacoes.\\

Falar que o comportamento do individuo é diferente que em grupo, realçr que a verdadeira democracia é valorizar o individuo acima do collectivo, que o contrario dá origen a maus resultados ex: Auchwitz, comportamentos feitos por individuos deploraveis por estarem em grupo na qual se o acto fosse individual não acontecia, e a desculpa "apenas fiz o que me mandaram", não é justificação teras de pagar o preço.\\


LEAN and Six Sigma.\\
\\
Best practices can come from national, say
the American National Standards Institute (ANSI) or the Canadian Standards
Association (CSA), or international, say ISO or Institute of Electrical and
Electronics Engineers (IEEE), standards organizations, professional associa-
tions, or consulting firms.\\

Eliminar desperdicio e resolução de problemas.\\

Confiança na liderança e opearadores.\\
Both organizational culture and a leader’s behaviors have been identified as critical
determinants of an organization’s effectiveness\\
Leaders have long been viewed as a primary influence on the creation of organizational
culture (e.g. Bennis and Nanus, 1985; Schein, 1983). According to Schein (1985), the “only
thing of real importance that leaders do is to create and manage culture”\\


A vida ensina e se não aprendemos ela ensiste e persiste até morrermos.\\

A liberdade é medida pela ethica e moralidade presente, quanto maior mais liberdade.\\


O universo é regido por leis fisicas, são visiveis e testaveis, tal como também existe leis de moral que esta entranhado no ser humano que vai além do dna não perceptivel, se saltares de uma montanha sofres as consequencias da gravidade e se ao atravesares as estrada e não olhares para ambos os lados para ver o transito corres o risco de ser atropelado, assim é a fisica e a moralidade.\\

Uma grande lissão que Linus Trovaldes nos ensina é que o respeito não nos é dado mas sim tem que ser conquitado, dai deduz-se que ao convivermos o respito começa do zero e vai se desenvolver tanto no sentido possitivo ou negativo e sera sempre reciproco.\\

Como nos ensina a regra de ouro, fazer ao proximo que queres que te seja feito a ti, também se pode traduzir que cada um recebe o que colhe, dai podemos afirmar que quem é do bem devemos retribuir com o bem e do mal, o mal, só assim é que no fim do dia o Bem vence sempre. O contrario é reconhecido pela verdadeira Injustiça.
Não esperes portanto o bem se fores pessoa do ruim, lol. Dai a complexidade de perceber as atitudes e comportamentos sem primeiro termos Empathia, e o mundo em geral, não vale apena atirarmos pedras se nossas telhas são feitas de vidro.\\
No organization today exists in a stable environment. Current scholars, especially the
proponents of complexity theories, consider that all organizations are under the influence
of multiple changes (Brown and Eisenhardt, 1997; Burnes, 2004a; Stacey et al., 2002;
Styhre, 2002; Tetenbaum, 1998). According to these scholars, change is inherent in
human action and therefore necessarily occurs in any context of human social interactions
(Ford and Ford, 1995). As organizations are sites of continuously evolving human action,
they are in a continuous state of change and, in order to survive, must develop the ability
to continuously change themselves (Burnes, 2004b; Tsoukas and Chia, 2002).\\

Constante mudança de adaptaçao ao meio ambiente.\\

the need for an organization to develop a marketing, strategy, or organizational culture.\\
the dimensions of organizational culture and an orientation towards individual and organizational change concerning the “acceptance” of a new system of values.\\

organizational change \\

Organizational culture determines how individuals behave, what people pay attention to,
how they respond to different situations, and how they socialize with new members and
exclude those who do not fit in (Spataro, 2005) \\

acknowledge that even in US and European companies, success rates are
not spectacular regarding efforts to change vision, values, and culture or business systems
and processes (Beer and Nohria, 2000; Beer et al., 1990; Carr et al., 1996).\\



\section{Cultura de uma Nação}
Quais as culturas possiveis das organizações na qual podem derivar do contexto onde se encontra, e quais não são possiveis.\\

Concentrar em Portugal, prós e contras.\\

\subsection{two}
something 2
\begin{enumerate}
\item Mission / Missão
\item Vission / Visão
\item core values / Valores
\item slogan / O lema desta casa é servir bem servir, dar saude e fazer sorrir.
\item company culture / cultura da organização
\end{enumerate}
\newpage
Gestão:\\ \\
\begin{minipage}{20cm}
\begin{minipage}{5cm}
Instrumentos
\begin{enumerate}
\item Planear
\item Organizar
\item Controlar\\ \\
\end{enumerate}
\end{minipage}
\begin{minipage}{5cm}
Funções
\begin{enumerate}
\item Liderança
\item Comuniacção
\item motivação
\item Tomada de decisão
\end{enumerate}
\end{minipage}
\end{minipage}


\newpage
\section{Conclusões}
The relationship between culture and leadership appears to be reciprocal—top leaders
create and maintain an organizational culture, which in turn influences the values, atti-
tudes, and behaviors of middle and entry-level leaders. Although leader–culture fit has not
been specifically studied in the published literature, we believe that there is value in exam-
ining the match between a leader’s behaviors and the culture in which they work. While
research hasn’t examined this at the level we discuss, current research does suggest that fit
is important at the national level and at the leader–follower level. Expansion of this
research will help determine what aspects of leader–culture fit are determinants of leader
and organizational effectiveness.
Although there are a variety of approaches that researchers can take to examining
leader–culture fit, we offer the following recommendations. First, although studies of
perceived fit are of limited value, the ease of collecting this data should motivate researchers
to start thinking about adding questions concerning leader–culture fit. Given the lack of
published findings, this research can begin shaping our knowledge about this phenomenon.
Second, while studies of subjective fit will be more important, researchers should utilize
360-degree measurement systems in order to also obtain the most objective fit indices
possible. This practice will likely tell us more about the impact of fit than just examining
leaders’ self-reports. Third, it is important to measure culture at the aggregate level in
order to ensure that the actual values of the organization are being captured, not just the
leader’s values. Although these recommendations may be difficult to achieve in practice,
they offer the best hope of leveraging leader–culture fit for the future.\\

\newpage
%%%%%%%%%%%%%%%%%%%%%%%%%%%%%%%%%%%%%%%%%%%%%%%%%%%%%%%%%%%%%%%%%%%%%%%%%%%%%%%%%%%%%
%\input{./input/EQUACAO}
%\input{./input/DEFINICAO}
%\newpage
%\input{Equacoes}
%%%%%%%%%%%%%%%%%%%%%%%%%%%%%%%%%%%%%%%%%%%%%%%%%%%%%%%%%%%%%%%%%%%%%%%%%%%%%%%%%%%%%%
%Figuras Bibliografia Index
\listoffigures
\cite{*}
\bibliography{./bibliography/Bibliography}
%outro metodo mas manual\input{Bibliografia}
%\printindex
\newpage
\footnote{Apontamento}
\end{document}
%%%%%%%%%%%%%%%%%%%%%%%%%%%%%%%%%%%%%%%%%%%%%%%%%%%%%%%%%%%%%%%%%%%%%%%%%%%%%%%%%%%%%%
